%PREAMBULO 

\documentclass[a4paper, titlepage]{report}%definizione obbligatoria tipo di documento con opzioni a scelta
\usepackage[T1]{fontenc}                %codifica caratteri e font utilizzati
\usepackage[utf8]{inputenc}             %codifica del protocollo
\usepackage[english, italian]{babel}    %lingue utilizzate, italiano quella principale perchè l'ultima delle elencate

\usepackage{esvect}
\usepackage{mathtools}
\usepackage{bm}
\usepackage{url}
\usepackage{emptypage}       %elimina le facciate blank
\usepackage{hyperref}        %definisce i collegamenti ipertestuali e Web
\usepackage{indentfirst}
\usepackage{amsmath}         %pacchetto per linguaggio matematico
\usepackage{amssymb}         %pacchetto per linguaggio matematico
\usepackage{amsthm} 
\usepackage{physics}         %tre stili predefiniti per gli enunciati

\usepackage{booktabs, caption} %pacchetti per le tabelle
\usepackage{graphicx} %pacchetto per le figure
\graphicspath{ {./images/} }
\usepackage{wrapfig}
%Definisco gli stili per  teoremi, definizioni, ecc

\theoremstyle{definition} %inizializzo definizioni                
\newtheorem{definizione}{Definizione}[chapter]  %[chapter]-->numerazione della def con riferimento al capitolo
\theoremstyle{plain}
\newtheorem{teorema}{Teorema}[chapter]
\theoremstyle{plain}
\newtheorem{proposizione}{Proposizione}[chapter]
\theoremstyle{remark}
\newtheorem*{osservazione}{Osservazione}
\theoremstyle{remark}
\newtheorem*{nota}{Nota}

\usepackage[vmargin=2.5cm,hmargin=2cm]{geometry}



\renewcommand{\v}[1]{\mathbf{\hat{#1}}} 
%%VERSORE
\renewcommand{\vec}[1]{\vv{\bm{#1}}}   
%%VETTORE BOLD E CON ARROW
\newcommand{\Norm}[1]{ \left\lVert {\vv{\bm{#1}}} \right\rVert}
%%%NORMA DI UN VETTORE


\begin{document}
    
\section{Vettori e Sistemi di riferimento}

    \subsection*{Le trasformazioni di Galileo}
Considerando due sistemi di riferimento, $S $ ed $S'$ in cui il secondo ha un moto 
di \emph{traslazione rettilinea uniforme} rispetto al primo, il vettore posizione e i principali
vettori cinematici si trasformano nel modo seguente:
\[
\begin{cases}
     \vec{r} = \vec{r}' + \vec{V}'  t' \\
     t = t'
\end{cases} 
\qquad 
\begin{cases}
     \vec{v} = \vec{v}' + \vec{V} \\
     \vec{a} = \vec{a}'
\end{cases}    
\]
Da queste trasformazioni si deduce che la velocità non è invariante e dipende dal sistema di 
riferimento:essa varia secondo la classica legge di composzione della velocità. \\
Al contrario l'accelerazione è invariante. \\
Il principio Galileiano afferma, quindi, che non è possibile distinguere due fenomeni fisici 


\subsection*{Teorema del triplo prodotto vettoriale}
\[
 \vec{a} \cross (\vec{b} \cross \vec{c}) = \vec{b} \cross (\vec{a} \cross \vec{c}) -
 \vec{c} \cross (\vec{a} \cross \vec{b})
\]
\subsection*{Derivate di vettori e versori }
\[
     \v{v} \equiv \frac{\vec{v}}{\Norm{v}}
\]
\emph{Derivata di un versore} (da mettere in sezione apposita)
Definendo un vettore $\omega = \v{u} \cross \v{n}$ con modulo $\frac{d \varphi}{dt}$
\[
 \frac{d\v{u}}{dt} = \frac{d \varphi}{dt} \v{n} = \vec{\omega} \cross \v{u}     
\]

\subsection*{Coordinate polari piane}
Un sistema di riferimento di coordinate polari piane è un sistema di coordinate 
caratterizzato da un \textbf{polo}, l'origine O, e da un \textbf{asse polare},
la semiretta orientata uscente da O. La direzione di ogni punto P è caratterizzata
da due versori:
\[
 \v{u}_r \quad (\text{versore radiale}) \qquad \text{e} \qquad \v{u}_{\theta}
 \quad  \text{(versore trasverso)}     
\]
con $\v{u}_r$ indicante la direzione del vettore direzione $\vec{r}$, che infatti si 
scrive come $\vec{r} = r \, \v{u}_r$, e con $\v{u}_{\theta}$ versore tangente alla 
circonferenza e diretto verso angoli $\theta$ crescenti ($\v{u}_r \perp \v{u}_{\theta}$).\\
Nel caso in cui la direzione dell'asse polare
sia quella del versore $\v{i}$ allora la relazione tra le coordinate polari piane (r, $\theta$)
e quelle cartesiane ($x,y$) è il seguente:
\[
\begin{cases}
     x = r cos \theta \\
     y = r sin \theta
\end{cases}
; \qquad
\begin{cases}
     r = \sqrt{x^2 + y^2} \\
     \theta = arctan \frac{y}{x}
\end{cases}     
\]
e tra i versori polari e cartesiani:
\[
\begin{cases}
     \v{u}_r = (\v{u}_r \cdot \v{i})\v{i} + (\v{u}_r \cdot \v{j})\v{j} =
     cos \theta\v{i} + sin \theta \v{j} \\
     \v{u}_{\theta}= (\v{u}_{\theta} \cdot \v{i})\v{i} + (\v{u}_{\theta} \cdot \v{j})\v{j} =
     -sin \theta \v{i} + cos \theta \v{j}
\end{cases}     
\]
Il vettore direzione $\vec{r} = \v{u}_r$ in relazione alle coordinate cartesiane:
\[
     \vec{r} = r cos \theta \v{i} + r sin \theta \v{j}     
\]
Mentre il vettore $\vec{\varphi}$ sempre tangente alla curva con direzione $\v{u}_{\theta}$
(in un moto circolare corrisponde alla velocità tangenziale): 
\[
    \vec{\varphi} = - \varphi sin \theta + \varphi cos \theta     
\]
\subsection*{Coordinate polari sferiche}


\section{Cinematica}
    
    \noindent Un corpo è in moto, \emph{rispetto ad un sistema di 
riferimento $S$}, quando la sua posizione in $S$ cambia
nel tempo. Le caratteristiche del moto del \emph{punto materiale}
sono note se è noto il vettore posizione $\vec{r}$ in funzione del
tempo, ovvero:
\[
    \vec{r}(t) = \begin{cases}
        x = x(t) \\
        y = y(t) \\
        z = z(t)
    \end{cases}
\]
nel sistema di riferimento $S$. 
Conoscere il vettore posizione significa conoscere come variano 
tutte le coordinate, $x, y, z$ in funzione del tempo. 
Nell'ipotesi implicita di continuità del moto(il tempo è una variabile
continua) il moto può essere descritto attraverso l' \textbf{equazione vettoriale}
\[
\vec{r} = \vec{r}(t)    
\]
L'insieme delle posizioni occupate dal punto nel suo moto è la sua
\textbf{traiettoria $\gamma$}. Nota la traiettoria $\gamma$, chiamiamo
con $s$ il numero reale detto \emph{ascissa curvilinea}, il cui modulo, $|s|$,
fornisce la lunghezza dell'arco di curva dall'origine scelta alla posizione
sulla traiettoria del punto P. Introducendo la variabile $s$ possiamo 
descrivere il moto anche con le seguenti due funzioni:
\[
    \begin{cases}
        \vec{r} = \vec{r}(s)    \\
        s = s(t)
    \end{cases}
\]
dove in un sistema di coordinate cartesiane, $\mathbb{R}^3$:
\[
    \vec{r} = \vec{r}(s)  \Longrightarrow 
    \begin{cases}
        x = x(t) \\
        y = y(t) \\
        z = z(t)
    \end{cases}
    \longrightarrow \text{equazione della traiettoria in forma parametrica}
\]
\[
   s= s(t) \longrightarrow \text{equazione oraria} 
\]
%%%%%%%%%%%%%%%%%%%%%%%%%%%%%%%
%     VELOCITA
%%%%%%%%%%%%%%%%%%%%%%%%%%

\subsection*{Il vettore velocità}

\noindent Si definisce \textbf{velocità media} il vettore:
\[
     \vec{v}_m \equiv \frac{\vec{r}(t + \Delta t)-\vec{r}(t)}{\Delta t}    
\]
Esso è indipendente dal percorso compiuto dal punto materiale nei vari istanti di tempo ma solo dalla posizione
inziale e finale.
Definiamo poi il vettore \textbf{velocità istantanea} nel modo seguente:
\[
\vec{v} \equiv \lim_{\Delta t \to 0} \vec{v}_m = \lim_{\Delta t \to 0} \frac{\Delta \vec{r}}{\Delta t}      
\]
\[
\vec{v}(t) \equiv \frac{d \vec{r}(t)}{dt}   
\]
Questa è la rappresentazione \emph{intrinseca} della velocità. Considerando due posizioni successive occuppate dal 
punto lungo la traiettoria $\gamma $, il vettore $\Delta \vec{r}$ ha direzione secante alle due posizioni sulla traiettoria. 
Quando l'arco di traiettoria $\Delta s$ tende a 0 allora:
\[
    \lim_{\Delta s \to 0} \frac{\Delta \vec{r}}{\Delta s} = 1
\]
In questa situazione il vettore velocità tende ad assumere direzione tangente alla traiettoria. A questo punto il 
\textbf{versore tangente} alla curva nel punto P sarà:
\[
   \v{u}_t = \lim_{\Delta s \to 0} \frac{\Delta \vec{r}}{\Delta s} = \frac{d \vec{r}}{ds}
\]
Questo versore rappresenta la direzione del vettore velocità istantea che può quindi essere riscritta come:
\[
    \vec{v}(t) \equiv \frac{d \vec{r}(t)}{dt} = \frac{d\vec{r}}{ds} \frac{ds}{dt} = \frac{ds}{dt} \v{u}
    = v_s \v{u}_t = \dot{s} \v{u}_t 
\]
\[
 \Norm{v} =  \abs{\frac{ds}{dt} }    \longrightarrow \text{modulo della velocità}
\]
La rappresentazione cartesiana della velocità è la seguente:
\[
 \vec{v} =  \frac{dx}{dt} \v{i} + \frac{dy}{dt} \v{j} + \frac{dz}{dt} \v{k}    
\]
\[
\vec{v} = \dot{x}\v{i} +  \dot{y}\v{j} +  \dot{z}\v{k}    
\]

\subsection*{Il vettore accelerazione}

Si definisce \textbf{accelerazione media} il vettore:
\[
    \vec{a}_m = \frac{\vec{v}(t + \Delta t)-\vec{v}(t)}{\Delta t}      
\]
Il vettore \textbf{accelerazione} istantanea è definito dal limite:
\[
    \vec{a} \equiv \lim_{\Delta t \to 0} \vec{a}_m = \lim_{\Delta t \to 0} \frac{\Delta \vec{v}}{\Delta t}      
\]
\[
 \vec{a}(t) = \frac{d \vec{v}}{dt} = \frac{d^2\vec{r}}{dt^2}    
\]
In coordinate cartesiane l'accelerazione corrisponde a:
\[
    \vec{a} =  \frac{dv_x}{dt} \v{i} + \frac{dv_y}{dt} \v{j} + \frac{dv_z}{dt} \v{k}
    = \frac{d^2x}{dt^2} \v{i} + \frac{d^2y}{dt^2} \v{j} + \frac{d^2z}{dt^2} \v{k}
\]
Dal calcolo di derivazione rispetto alla velocità:
\[
    \vec{a} = \frac{d \vec{v}}{dt} = \frac{d}{dt} (v_s \v{u}_t) = \frac{dv_s}{dt}\v{u}_t
    + \frac{d\v{u}_t}{dt}
\]
si nota come l'accelerazione sia composta da due contributi: uno parrallelo a $\v{u}_t$ e uno normale a 
$\v{u}_t$:
\[
 \vec{a} = \vec{a}_t + \vec{a}_n   
\]
\[
 \vec{a}_t = \frac{d^2 s}{dt^2} \v{u}_t \equiv \ddot{s} \v{u}_t \quad (\textbf{accelerazione tangenziale})
\]
\[
    \vec{a}_n = \frac{d\v{u}_t}{dt} = \frac{d\v{u}_t}{ds} \frac{ds}{dt} = \dot{s}\frac{d\v{u}_t}{ds}
    = \dot{s}\frac{d \varphi}{ds}\v{u}_n = \frac{\dot{s}^2}{\rho}\v{u}_n \quad (\textbf{accelerazione normale})    
\]
Gli ultimi due passaggi sono ottenuti derivando il versore $\v{u}_t$ secondo il metodo noto trovando quindi una
forma che è caratteristica della traiettoria $\gamma$ in esame(vedi moto circolare). \\
Si deduce quindi che qualsiasi moto con traiettoria curva è \emph{accelerato} dal momento che la componente normale
dell'accelerazione $\v{a}_n$ non è nulla(come nel caso di un moto rettilineo).
\subsection*{Moto circolare uniforme}
\label{cinem: circ-unif}

\section{Dinamica}

Esistono due categorie principali di forze
\begin{itemize}
    \item Le forze di contatto
    \item Le forze con azione a distanza(per esempio la forza centrale)
\end{itemize}
\textbf{TRASFORMAZIONI DI GALILEO} \\
Considerando due sistemi di riferimento, $S $ ed $S'$ in cui il secondo ha un moto 
di \emph{traslazione rettilinea uniforme} rispetto al primo, il vettore posizione e i principali
vettori cinematici si trasformano nel modo seguente:
\[
\begin{cases}
     \vec{r} = \vec{r}' + \vec{V}'  t' \\
     t = t'
\end{cases} 
\qquad 
\begin{cases}
     \vec{v} = \vec{v}' + \vec{V} \\
     \vec{a} = \vec{a}'
\end{cases}    
\]
Da queste trasformazioni si deduce che la velocità non è invariante e dipende dal sistema di 
riferimento:essa varia secondo la classica legge di composzione della velocità. \\
Al contrario l'accelerazione è invariante. \\
Il principio Galileiano afferma, quindi, che non è possibile distinguere due fenomeni fisici. 

Per quanto riguarda le leggi di Newton queste hanno come contesto di validità quello dei sistemi di
riferimento \textbf{inerziali}, altrimenti devono subire delle modifiche.
Per verificare che un sistema di riferimento è inerziale bisogna individuare almeno un \emph{corpo
libero}, ovvero isolato, non soggetto a forze esterne, oppure tale che la risultante agente su
questo sia nulla: se questo corpo è in quiete oppure si muove con velocità costante(in un moto rettilineo)
allora quello in esame è un sistema di riferimento inerziale.
Una volta assicurata l'esistenza di un sistema di riferimento inerziale allora si può affermare l'esistenza
di infiniti sistemi di riferimento di questo tipo in quanto, grazie al principio di relatività galileaiana,
è noto che qualsiasi sistema $S'$ che stia traslando rispetto ad $S$ a velocità costante ed in linea retta
costituisce un nuovo sistema di riferimento inerziale.

\begin{legge}[Prima Legge della dinamica]
    Esiste almeno un sistema di riferimento (inerziale) rispetto al quale ogni punto 
    materiale libero ha velocità costante
\end{legge}
La prima legge della dinamica assicura l'esistenza non solo di un sistema di riferimento 
in cui un corpo inizialmente in quiete resta in quiete, ma di infiniti sistemi, ottenuti
attraverso una traslazione


\begin{legge}[Seconda legge della dinamica]
    In un sistema di riferimento inerziale , l'accelerazione di un corpo è sempre dovuta all'azione
    di forze; tra forza risultante ed accelerazione sussiste, in ogni istante, la relazione
    \[
        \vec{F}(t) = m \vec{a}(t)     
    \]

    
\end{legge}


\textbf{IL PENDOLO SEMPLICE} \\
Si tratta di un punto materiale di massa $m$ attaccato ad un sostegno rigido tramite un filo \emph{insestensibile}
e di massa trascurabile, di lunghezza $l$. Viene poi trascurato ogni attrito dovuto all'aria o al filo. \\
Le forze in gioco sono le seguenti:
\[
 \vec{F}_p + \vec{R} = m \vec{a}    
\]
Il sistema che descrive le relazioni scalari delle forze in gioco:
\[
\begin{cases}
    -mg sin\theta = m \ddot{s} \\
    -mg cos \theta + R = m \frac{\dot{s}^2}{L}
\end{cases}    
\]
Utilizzando il fatto che $\theta = \frac{s}{L}$ la prima equazione 
diventa:
\[
\ddot{s} + g sin(\frac{s}{L}) = 0    
\]
Essendo questa un'equazione differenziale trascendente senza soluzione
analitica, operiamo l'approssimazione 
\[
    sin(\frac{s}{L}) \approx \frac{s}{L}
\]
sviluppando la serie di Taylor al primo ordine(la percentuale di errore 
è molto piccola per angoli piccoli). A questo punto si ottiene l'equazione
\[
    \ddot{s} + \frac{g}{L}s = 0
\]
che è la nota equazione del moto oscillatorio armonico con legge oraria:
\[
    s(t) = Acos (\omega_0 t + \varphi_0)
\]
e di pulsazione e periodo:
\[
 \omega_0 = \sqrt{\frac{g}{L}}   \qquad 
 T = \frac{2 \pi}{\omega_0} = 2\pi \sqrt{\frac{L}{g}} 
\]
La legge orari
\subsection*{Applicazione delle leggi della dinamica}
\textbf{LE FORZE ELASTICHE} \\
Le forze esercitate dalle molle sono un'esempio di forze che dipendono solo dalla posizione. Nel caso
di una molla ideale si ha che la forza di richiamo agisce solo sull'asse della molla, in verso
 opposto alla deformazione di questa ed è proporzionale all'allungamento.
Ciò viene descritto attraverso la relazione empirica seguente.
\begin{legge}[Legge di Hooke]
    \[
    \vec{F} = -kx \, \v{i}
    \]
\end{legge} 
Analizziamo il moto di una molla ideale. \\
Applicando il secondo principio della dinamica:
\[
  m \ddot{x} = -\frac{k}{m} x    
\]
Dalla risoluzione dell'equazione differenziale caratterizzante il moto oscillatorio armonico si ha:
\[
x(t) = x_0 cos(\omega_0 t + \varphi_0)    
\]
da cui:
\[
 \ddot{x}(t) = - \omega^2 x_0 cos(\omega_0 t)    
\]
Da cui si ottiene:
\[
  \omega_0 = \sqrt{\frac{k}{m}}    
\]


\end{document}