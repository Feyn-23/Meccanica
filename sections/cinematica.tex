Un corpo è in moto, \emph{rispetto ad un sistema di 
riferimento $S$}, quando la sua posizione in $S$ cambia
nel tempo. Le caratteristiche del moto del \emph{punto materiale}
sono note se è noto il vettore posizione $\vec{r}$ in funzione del
tempo, ovvero:
\[
    \vec{r}(t) = \begin{cases}
        x = x(t) \\
        y = y(t) \\
        z = z(t)
    \end{cases}
\]
nel sistema di riferimento $S$. 
Conoscere il vettore posizione significa conoscere come variano 
tutte le coordinate, $x, y, z$ in funzione del tempo. 
Nell'ipotesi implicita di continuità del moto(il tempo è una variabile
continua) il moto può essere descritto attraverso l' \textbf{equazione vettoriale}
\[
\vec{r} = \vec{r}(t)    
\]
L'insieme delle posizioni occupate dal punto nel suo moto è la sua
\textbf{traiettoria $\gamma$}. Nota la traiettoria $\gamma$, chiamiamo
con $s$ il numero reale detto ascissa curvilinea, il cui modulo, $|s|$,
fornisce la lunghezza dell'arco di curva dall'origine scelta alla posizione
sulla traiettoria del punto P. Introducendo la variabile $s$ possiamo 
descrivere il moto anche con le seguenti due funzioni:
\[
    \begin{cases}
        \vec{r} = \vec{r}(s)    \\
        s = s(t)
    \end{cases}
\]
dove in un sistema di coordinate cartesiane, $\mathbb{R}^3$:
\[
    \vec{r} = \vec{r}(s)  \Longrightarrow 
    \begin{cases}
        x = x(t) \\
        y = y(t) \\
        z = z(t)
    \end{cases}
    \longrightarrow \text{equazione della traiettoria in forma parametrica}
\]
\[
   s= s(t) \longrightarrow \text{equazione oraria} 
\]