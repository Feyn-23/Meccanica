\noindent Un corpo è in moto, \emph{rispetto ad un sistema di 
riferimento $S$}, quando la sua posizione in $S$ cambia
nel tempo. Le caratteristiche del moto del \emph{punto materiale}
sono note se è noto il vettore posizione $\vec{r}$ in funzione del
tempo, ovvero:
\[
    \vec{r}(t) = \begin{cases}
        x = x(t) \\
        y = y(t) \\
        z = z(t)
    \end{cases}
\]
nel sistema di riferimento $S$. 
Conoscere il vettore posizione significa conoscere come variano 
tutte le coordinate, $x, y, z$ in funzione del tempo. 
Nell'ipotesi implicita di continuità del moto(il tempo è una variabile
continua) il moto può essere descritto attraverso l' \textbf{equazione vettoriale}
\[
\vec{r} = \vec{r}(t)    
\]
L'insieme delle posizioni occupate dal punto nel suo moto è la sua
\textbf{traiettoria $\gamma$}. Nota la traiettoria $\gamma$, chiamiamo
con $s$ il numero reale detto \emph{ascissa curvilinea}, il cui modulo, $|s|$,
fornisce la lunghezza dell'arco di curva dall'origine scelta alla posizione
sulla traiettoria del punto P. Introducendo la variabile $s$ possiamo 
descrivere il moto anche con le seguenti due funzioni:
\[
    \begin{cases}
        \vec{r} = \vec{r}(s)    \\
        s = s(t)
    \end{cases}
\]
dove in un sistema di coordinate cartesiane, $\mathbb{R}^3$:
\[
    \vec{r} = \vec{r}(s)  \Longrightarrow 
    \begin{cases}
        x = x(t) \\
        y = y(t) \\
        z = z(t)
    \end{cases}
    \longrightarrow \text{equazione della traiettoria in forma parametrica}
\]
\[
   s= s(t) \longrightarrow \text{equazione oraria} 
\]
%%%%%%%%%%%%%%%%%%%%%%%%%%%%%%%
%     VELOCITA
%%%%%%%%%%%%%%%%%%%%%%%%%%

\subsection*{Il vettore velocità}

\noindent Si definisce \textbf{velocità media} il vettore:
\[
     \vec{v}_m \equiv \frac{\vec{r}(t + \Delta t)-\vec{r}(t)}{\Delta t}    
\]
Esso è indipendente dal percorso compiuto dal punto materiale nei vari istanti di tempo ma solo dalla posizione
inziale e finale.
Definiamo poi il vettore \textbf{velocità istantanea} nel modo seguente:
\[
\vec{v} \equiv \lim_{\Delta t \to 0} \vec{v}_m = \lim_{\Delta t \to 0} \frac{\Delta \vec{r}}{\Delta t}      
\]
\[
\vec{v}(t) \equiv \frac{d \vec{r}(t)}{dt}   
\]
Questa è la rappresentazione \emph{intrinseca} della velocità. Considerando due posizioni successive occuppate dal 
punto lungo la traiettoria $\gamma $, il vettore $\Delta \vec{r}$ ha direzione secante alle due posizioni sulla traiettoria. 
Quando l'arco di traiettoria $\Delta s$ tende a 0 allora:
\[
    \lim_{\Delta s \to 0} \frac{\Delta \vec{r}}{\Delta s} = 1
\]
In questa situazione il vettore velocità tende ad assumere direzione tangente alla traiettoria. A questo punto il 
\textbf{versore tangente} alla curva nel punto P sarà:
\[
   \v{u}_t = \lim_{\Delta s \to 0} \frac{\Delta \vec{r}}{\Delta s} = \frac{d \vec{r}}{ds}
\]
Questo versore rappresenta la direzione del vettore velocità istantea che può quindi essere riscritta come:
\[
    \vec{v}(t) \equiv \frac{d \vec{r}(t)}{dt} = \frac{d\vec{r}}{ds} \frac{ds}{dt} = \frac{ds}{dt} \v{u}
    = v_s \v{u}_t = \dot{s} \v{u}_t 
\]
\[
 \Norm{v} =  \abs{\frac{ds}{dt} }    \longrightarrow \text{modulo della velocità}
\]
La rappresentazione cartesiana della velocità è la seguente:
\[
 \vec{v} =  \frac{dx}{dt} \v{i} + \frac{dy}{dt} \v{j} + \frac{dz}{dt} \v{k}    
\]
\[
\vec{v} = \dot{x}\v{i} +  \dot{y}\v{j} +  \dot{z}\v{k}    
\]
%%%%%%%%%%%%%%%%
%%%ACCELERAZIONE
%%%%%%%%%%%%%%%%
\subsection*{Il vettore accelerazione}

Si definisce \textbf{accelerazione media} il vettore:
\[
    \vec{a}_m = \frac{\vec{v}(t + \Delta t)-\vec{v}(t)}{\Delta t}      
\]
Il vettore \textbf{accelerazione} istantanea è definito dal limite:
\[
    \vec{a} \equiv \lim_{\Delta t \to 0} \vec{a}_m = \lim_{\Delta t \to 0} \frac{\Delta \vec{v}}{\Delta t}      
\]
\[
 \vec{a}(t) = \frac{d \vec{v}}{dt} = \frac{d^2\vec{r}}{dt^2}    
\]
In coordinate cartesiane l'accelerazione corrisponde a:
\[
    \vec{a} =  \frac{dv_x}{dt} \v{i} + \frac{dv_y}{dt} \v{j} + \frac{dv_z}{dt} \v{k}
    = \frac{d^2x}{dt^2} \v{i} + \frac{d^2y}{dt^2} \v{j} + \frac{d^2z}{dt^2} \v{k}
\]
Dal calcolo di derivazione rispetto alla velocità:
\[
    \vec{a} = \frac{d \vec{v}}{dt} = \frac{d}{dt} (v_s \v{u}_t) = \frac{dv_s}{dt}\v{u}_t
    + \frac{d\v{u}_t}{dt}
\]
si nota come l'accelerazione sia composta da due contributi: uno parrallelo a $\v{u}_t$ e uno normale a 
$\v{u}_t$:
\[
 \vec{a} = \vec{a}_t + \vec{a}_n   
\]
\[
 \vec{a}_t = \frac{d^2 s}{dt^2} \v{u}_t \equiv \ddot{s} \v{u}_t \quad (\textbf{accelerazione tangenziale})
\]
\[
    \vec{a}_n = \frac{d\v{u}_t}{dt} = \frac{d\v{u}_t}{ds} \frac{ds}{dt} = \dot{s}\frac{d\v{u}_t}{ds}
    = \dot{s}\frac{d \varphi}{ds}\v{u}_n = \frac{\dot{s}^2}{\rho}\v{u}_n \quad (\textbf{accelerazione normale})    
\]
Gli ultimi due passaggi sono ottenuti derivando il versore $\v{u}_t$ secondo il metodo noto trovando quindi una
forma che è caratteristica della traiettoria $\gamma$ in esame(vedi moto circolare). \\
Si deduce quindi che qualsiasi moto con traiettoria curva è \emph{accelerato} dal momento che la componente normale
dell'accelerazione $\v{a}_n$ non è nulla(come nel caso di un moto rettilineo).






%%%%%%%%%%%%%%%%%%%
%%%MOTI%%%%%%%
%%%%%%%%%%%%%%%%%%
\subsection*{Moto rettilineo uniforme}
\subsection*{Moto rettilineo uniformemente accelerato}
\subsection*{Moto circolare uniforme}
\label{cinem: circ-unif}