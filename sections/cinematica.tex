\noindent Un corpo è in moto, \emph{rispetto ad un sistema di 
riferimento $S$}, quando la sua posizione in $S$ cambia
nel tempo. Le caratteristiche del moto del \emph{punto materiale}
sono note se è noto il vettore posizione $\vec{r}$ in funzione del
tempo, ovvero:
\[
    \vec{r}(t) = \begin{cases}
        x = x(t) \\
        y = y(t) \\
        z = z(t)
    \end{cases}
\]
nel sistema di riferimento $S$. 
Conoscere il vettore posizione significa conoscere come variano 
tutte le coordinate, $x, y, z$ in funzione del tempo. 
Nell'ipotesi implicita di continuità del moto(il tempo è una variabile
continua) il moto può essere descritto attraverso l' \textbf{equazione vettoriale}
\[
\vec{r} = \vec{r}(t)    
\]
L'insieme delle posizioni occupate dal punto nel suo moto è la sua
\textbf{traiettoria $\gamma$}. Nota la traiettoria $\gamma$, chiamiamo
con $s$ il numero reale detto \emph{ascissa curvilinea}, il cui modulo, $|s|$,
fornisce la lunghezza dell'arco di curva dall'origine scelta alla posizione
sulla traiettoria del punto P. Introducendo la variabile $s$ possiamo 
descrivere il moto anche con le seguenti due funzioni:
\[
    \begin{cases}
        \vec{r} = \vec{r}(s)    \\
        s = s(t)
    \end{cases}
\]
dove in un sistema di coordinate cartesiane, $\mathbb{R}^3$:
\[
    \vec{r} = \vec{r}(s)  \Longrightarrow 
    \begin{cases}
        x = x(t) \\
        y = y(t) \\
        z = z(t)
    \end{cases}
    \longrightarrow \text{equazione della traiettoria in forma parametrica}
\]
\[
   s= s(t) \longrightarrow \text{equazione oraria} 
\]
%%%%%%%%%%%%%%%%%%%%%%%%%%%%%%%
%     VELOCITA
%%%%%%%%%%%%%%%%%%%%%%%%%%

\subsection*{Il vettore velocità}

\noindent Si definisce \textbf{velocità media} il vettore:
\[
     \vec{v}_m \equiv \frac{\vec{r}(t + \Delta t)-\vec{r}(t)}{\Delta t}    
\]
Esso è indipendente dal percorso compiuto dal punto materiale nei vari istanti di tempo ma solo dalla posizione
inziale e finale.
Definiamo poi il vettore \textbf{velocità istantanea} nel modo seguente:
\[
\vec{v} \equiv \lim_{\Delta t \to 0} \vec{v}_m = \lim_{\Delta t \to 0} \frac{\Delta \vec{r}}{\Delta t}      
\]
\[
\vec{v}(t) \equiv \frac{d \vec{r}(t)}{dt}   
\]
Questa è la rappresentazione \emph{intrinseca} della velocità. Considerando due posizioni successive occuppate dal 
punto lungo la traiettoria $\gamma $, il vettore $\Delta \vec{r}$ ha direzione secante alle due posizioni sulla traiettoria. 
Quando l'arco di traiettoria $\Delta s$ tende a 0 allora:
\[
    \lim_{\Delta s \to 0} \frac{\Delta \vec{r}}{\Delta s} = 1
\]
In questa situazione il vettore velocità tende ad assumere direzione tangente alla traiettoria. A questo punto il 
\textbf{versore tangente} alla curva nel punto P sarà:
\[
   \v{u}_t = \lim_{\Delta s \to 0} \frac{\Delta \vec{r}}{\Delta s} = \frac{d \vec{r}}{ds}
\]
Questo versore rappresenta la direzione del vettore velocità istantea che può quindi essere riscritta come:
\[
    \vec{v}(t) \equiv \frac{d \vec{r}(t)}{dt} = \frac{d\vec{r}}{ds} \frac{ds}{dt} = \frac{ds}{dt} \v{u}
    = v_s \v{u}_t = \dot{s} \v{u}_t 
\]
\[
 v_s =  \frac{ds}{dt}     \longrightarrow \text{modulo della velocità}
\]
La rappresentazione cartesiana della velocità è la seguente:
\[
 \vec{v} =  \frac{dx}{dt} \v{i} + \frac{dy}{dt} \v{j} + \frac{dz}{dt} \v{k}    
\]
\[
\vec{v} = \dot{x}\v{i} +  \dot{y}\v{j} +  \dot{z}\v{k}    
\]
%%%%%%%%%%%%%%%%
%%%ACCELERAZIONE
%%%%%%%%%%%%%%%%
\subsection*{Il vettore accelerazione}

Si definisce \textbf{accelerazione media} il vettore:
\[
    \vec{a}_m = \frac{\vec{v}(t + \Delta t)-\vec{v}(t)}{\Delta t}      
\]
Il vettore \textbf{accelerazione} istantanea è definito dal limite:
\[
    \vec{a} \equiv \lim_{\Delta t \to 0} \vec{a}_m = \lim_{\Delta t \to 0} \frac{\Delta \vec{v}}{\Delta t}      
\]
\[
 \vec{a}(t) = \frac{d \vec{v}}{dt} = \frac{d^2\vec{r}}{dt^2}    
\]
In coordinate cartesiane l'accelerazione corrisponde a:
\[
    \vec{a} =  \frac{dv_x}{dt} \v{i} + \frac{dv_y}{dt} \v{j} + \frac{dv_z}{dt} \v{k}
    = \frac{d^2x}{dt^2} \v{i} + \frac{d^2y}{dt^2} \v{j} + \frac{d^2z}{dt^2} \v{k}
\]
Dal calcolo di derivazione rispetto alla velocità:
\[
    \vec{a} = \frac{d \vec{v}}{dt} = \frac{d}{dt} (v_s \v{u}_t) = \frac{dv_s}{dt}\v{u}_t
    + \frac{d\v{u}_t}{dt}
\]
si nota come l'accelerazione sia composta da due contributi: uno parrallelo a $\v{u}_t$ e uno normale a 
$\v{u}_t$:
\[
 \vec{a} = \vec{a}_t + \vec{a}_n   
\]
\[
 \vec{a}_t = \frac{d^2 s}{dt^2} \v{u}_t \equiv \ddot{s} \v{u}_t \qquad (\textbf{accelerazione tangenziale})
\]
\[
    \vec{a}_n = \frac{d\v{u}_t}{dt} = \frac{d\v{u}_t}{ds} \frac{ds}{dt} = \dot{s}\frac{d\v{u}_t}{ds}
    = \dot{s}\frac{d \varphi}{ds}\v{u}_n = \frac{\dot{s}^2}{\rho}\v{u}_n \quad (\textbf{accelerazione normale})    
\]
dove $\rho$ è il raggio della \emph{circonferenza osculatrice}, ovvero la circonferenza che approssima localmente il 
moto di un punto materiale, $\v{u}_n$ è il versore $\perp$ a $\v{u}_n$. 
Gli ultimi due passaggi sono ottenuti derivando il versore $\v{u}_t$ secondo il metodo noto trovando quindi una
forma che è caratteristica della traiettoria $\gamma$ in esame. \\
Si deduce quindi che qualsiasi moto con traiettoria curva è \emph{accelerato} dal momento che la componente normale
dell'accelerazione $\v{a}_n$ non è nulla(come nel caso di un moto rettilineo $\longrightarrow \rho = cost$).


%%%%%%%%%%%%%%%%%%%
%%%MOTI%%%%%%%
%%%%%%%%%%%%%%%%%%
I moti piu elementari di cui è nota la traiettoria $\gamma$ possono essere descritti tramite le due equazioni:
\[
  \begin{cases}
      \vec{v} = \dot{s}\v{u}_t \\ 
      \vec{a} = \ddot{s}\v{u}_t + \frac{\dot{s}^2}{\rho}\v{u}_n
  \end{cases}  
\]








\subsection*{Moti elementari: moto rettilineo uniforme}
Determinare  l'equazione del moto significa conoscendo le espressioni di  velocità e accelerazione e nota la traiettoria 
$\gamma $ significa risolvere il \emph{problema inverso} della cinematica.
In un moto rettilineo uniforme il punto materiale percorre spazi uguale in tempi uguali: si ha $\dot{s}(t) = \dot{s}_0
\equiv v_0 = cost$. \\
Per determinare l'espressione di $s(t)$:
\[
s(t) = \int \dot{s} dt = \dot{s}_0 \int dt = \dot{s}_0 t + C = v_0 t + C
\]
Sapendo quanto vale $s(t_0)= s_0$ si ha che la \textbf{legge oraria} è :
\[
   s = \dot{s_0}t + s_0 = v_0 t + s_0    
\]
In coordinate cartesiane:
\[
  \vec{v} = v_0 \v{i}     
\]
Dato che il moto avviene solo su un asse, la legge oraria è:
\[
  x(t) = v_0 t + x_0    
\]
L'accelerazione in ogni istante $\vec{a}(t) = 0$. 




\subsection*{Moti elementari: moto rettilineo uniformemente accelerato}

Questo tipo di moto è caratterizzato da un'accelerazione costante ovvero $\ddot{s}(t)
= \frac{d\dot{s}}{dt} \equiv a(t) = cost$. \\
Attraverso l'integrazione come per il moto rettilineo uniforme, e preso $t(0)= 0$:
\[
  \dot{s}= \ddot{s}_0t + \dot{s}_0    
\]
si ottiene l'equazione differenziale che descrive il cambiamento nel tempo della 
velocità che, risolta, fornisce l'espressione della funzione di $s(t)$:
\[
s = \frac{1}{2}\ddot{s_0}t^2 + \dot{s_0}t + s_0f    
\]
Nel caso in cui $t_0 \neq 0$ risolvendo il sistema dato dalle due relazioni precedenti considerando 
un tempo $(t-t_0)$:
\[
\begin{cases}

    \dot{s}= \ddot{s}_0(t-t_0) + \dot{s}_0    \\
    s = \frac{1}{2}\ddot{s_0}(t-t_0)^2 + \dot{s_0}(t-t_0) + s_0   

\end{cases}    
\]
si ottiene l'utile relazione:
\[
v^2 = {v_0}^2 + 2 \ddot{s_0}(s-s_0) \equiv {v_0}^2 + 2 a_t(s-s_0)   
\]
Analogamente all'altro moto rettilineo trattato, il moto avviene solo su un asse, quindi le leggi
che descrivono il moto sono:
\[
    x(t) = \frac{1}{2}a_0t^2 + v_0t + x_0   \qquad \textbf{(legge oraria)}
\]
\[
   \dot{x}(t) = a_0t + v_0 \quad \textbf{(variazione di velocità)}
\]


\subsection*{Moti elementari: moto circolare}
\label{cinem: circ-unif}
Descrivendo la traiettoria di un moto circolare una circonferenza, questa avrà 
un'equazione del tipo:
\[
  x^2 + y^2 = R^2    
\] 
Fissando un sistema di riferimento ottimale a questa situazione, ovvero un sistema
cartesiano rappresentante il piano $xy$ con $z=0$ con origine nel centro della traiettoria
circolare. Vi è una relazione fissa tra arco di circonfernza $s$ percorso e l'angolo $\theta$
relativo:
\[
s = r \cdot \theta     
\]
Nel caso moto circolare uniforme, la condizione di uniformità consiste nel fatto che il punto
materiale percorre angoli uguali in tempi uguali ovvero:
\[
   \dot{\theta} = cost  
\]
In particolare questa costante è definita \textbf{velocità angolare} del punto materiale:
\[
 \omega \equiv \frac{d \theta}{dt} = cost 
\]
e, essendo costante, vale in ogni punto la relazione $\theta = \omega t$.\\
Considerando il \textbf{periodo $T$} di rotazione, ovvero il tempo necessario per percorrere un giro completo
della traiettoria(ritornando al punto di partenza), la velocità angolare vale:
\[
   \omega = \frac{2 \pi}{T}  
\]
Sapendo che la frequenza $f$ è l'inverso del periodo, $f \equiv \frac{1}{T}$ allora:
\[
    \omega = 2 \pi f    
\]
A questo punto possiamo definire l'\textbf{accelerazione angolare} del punto come:
\[
   \alpha \equiv \ddot{\theta} \equiv \frac{d\omega}{dt}     
\]
La condizione di uniformità può essere rappresentata con $\alpha = 0$ in quanto ciò è equivalente a 
$\omega = cost$. 
Durante il moto il vettore direzione $\vec{r}$ ha modulo costante coincidente con il raggio 
$R$ e direzione data dal versore radiale $\v{u}_r$:
\[
 \vec{r} = r \v{u}_r = r cos \theta \v{i} + r sin \theta \v{j} = r cos (\omega t) \v{i} +
 r sin (\omega t) \v{j}    
\]
Andiamo a definire il vettore velocità $\vec{v}$:
\[
    \vec{v} \equiv \frac{d \vec{r}}{dt} = 
    \begin{cases}
        \frac{dx}{dt} = -r \omega sin (\omega t) \\
        \frac{dy}{dt} = r \omega cos (\omega t) 
    \end{cases}
\]
\[
 \vec{v} = r \omega(-sin(\omega t)\v{i} +cos (\omega t)\v{j}) 
\]
Il versore che definisce la direzione di $\vec{v}$ è:
\[
   \v{u}_{\theta} \equiv \frac{\vec{v}}{\Norm{v}} = \frac{\vec{v}}{r \omega} =  
   -sin(\omega t)\v{i} +cos (\omega t)\v{j}   
\]
Da qui si ricava un'importante risultato, ovvero il fatto che la velocità è sempre tangente alla traiettoria,
è quindi una \textbf{velocità tangenziale}. Ciò è dimostrato dalla perpendicolarità di $\v{u}_r$ e $\v{u}_{\theta}$
in ogni punto:
\[
    \v{u}_r \cdot \v{u}_{\theta} = -cos(\omega t)sin (\omega t) + cos(\omega t)sin (\omega t) = 0
\]
Introducendo il vettore $\vec{\omega} = \omega \v{k}$ (uscente dal piano) che ha il modulo della velocità
angolare posso definire intrinsecamente il vettore velocità nei moti circolari:
\[
\vec{v}(t) = \vec{\omega}(t) \cross \vec{r}(t)    
\]
Andiamo a questo punto a trovare l'espressione del vettore accelerazione:
\[
\vec{a} \equiv \frac{d \vec{v}}{dt} = 
\vec{v} \equiv \frac{d \vec{r}}{dt} = 
\begin{cases}
    \frac{dv_x}{dt} = -r \omega^2 cos (\omega t) \\
    \frac{dv_y}{dt} = -r \omega^2 sin (\omega t) 
\end{cases}
\]
\[
   \vec{a} = -r \omega^2(cos(\omega t)\v{i} +sin (\omega t)\v{j}) 
\]
\[
   \vec{a} = -r \omega^2 \, \v{u}_r    
\]
Da cui notiamo che l'accelerazione ha verso opposto a quello del versore radiale e quindi è una 
\textbf{accelerazione centripeta}. \\
Sfruttando la definizione intrinseca di velocità attraveso un semplice calcolo, sfruttando il teorema
del prodotto triplo, si arriva alla definzione intrinseca di accelerazione in un moto circolare:
\[
  \vec{a} =\underbrace{\alpha \cdot r \cdot \v{u}_{\theta} }_{\text{tangenziale}}- \underbrace{\omega^2 
  \cdot r \cdot \v{u}_r}_{\text{centripeta}}    
\]
