Esistono due categorie principali di forze
\begin{itemize}
    \item Le forze di contatto
    \item Le forze con azione a distanza(per esempio la forza centrale)
\end{itemize}
\textbf{TRASFORMAZIONI DI GALILEO} \\
Considerando due sistemi di riferimento, $S $ ed $S'$ in cui il secondo ha un moto 
di \emph{traslazione rettilinea uniforme} rispetto al primo, il vettore posizione e i principali
vettori cinematici si trasformano nel modo seguente:
\[
\begin{cases}
     \vec{r} = \vec{r}' + \vec{V}'  t' \\
     t = t'
\end{cases} 
\qquad 
\begin{cases}
     \vec{v} = \vec{v}' + \vec{V} \\
     \vec{a} = \vec{a}'
\end{cases}    
\]
Da queste trasformazioni si deduce che la velocità non è invariante e dipende dal sistema di 
riferimento:essa varia secondo la classica legge di composzione della velocità. \\
Al contrario l'accelerazione è invariante. \\
Il principio Galileiano afferma, quindi, che non è possibile distinguere due fenomeni fisici. 

Per quanto riguarda le leggi di Newton queste hanno come contesto di validità quello dei sistemi di
riferimento \textbf{inerziali}, altrimenti devono subire delle modifiche.
Per verificare che un sistema di riferimento è inerziale bisogna individuare almeno un \emph{corpo
libero}, ovvero isolato, non soggetto a forze esterne, oppure tale che la risultante agente su
questo sia nulla: se questo corpo è in quiete oppure si muove con velocità costante(in un moto rettilineo)
allora quello in esame è un sistema di riferimento inerziale.
Una volta assicurata l'esistenza di un sistema di riferimento inerziale allora si può affermare l'esistenza
di infiniti sistemi di riferimento di questo tipo in quanto, grazie al principio di relatività galileaiana,
è noto che qualsiasi sistema $S'$ che stia traslando rispetto ad $S$ a velocità costante ed in linea retta
costituisce un nuovo sistema di riferimento inerziale.

\begin{legge}[Prima Legge della dinamica]
    Esiste almeno un sistema di riferimento (inerziale) rispetto al quale ogni punto 
    materiale libero ha velocità costante
\end{legge}
La prima legge della dinamica assicura l'esistenza non solo di un sistema di riferimento 
in cui un corpo inizialmente in quiete resta in quiete, ma di infiniti sistemi, ottenuti
attraverso una traslazione


\begin{legge}[Seconda legge della dinamica]
    In un sistema di riferimento inerziale , l'accelerazione di un corpo è sempre dovuta all'azione
    di forze; tra forza risultante ed accelerazione sussiste, in ogni istante, la relazione
    \[
        \vec{F}(t) = m \vec{a}(t)     
    \]

    
\end{legge}


\textbf{IL PENDOLO SEMPLICE} \\
Si tratta di un punto materiale di massa $m$ attaccato ad un sostegno rigido tramite un filo \emph{insestensibile}
e di massa trascurabile, di lunghezza $l$. Viene poi trascurato ogni attrito dovuto all'aria o al filo. \\
Le forze in gioco sono le seguenti:
\[
 \vec{F}_p + \vec{R} = m \vec{a}    
\]
Il sistema che descrive le relazioni scalari delle forze in gioco:
\[
\begin{cases}
    -mg sin\theta = m \ddot{s} \\
    -mg cos \theta + R = m \frac{\dot{s}^2}{L}
\end{cases}    
\]
Utilizzando il fatto che $\theta = \frac{s}{L}$ la prima equazione 
diventa:
\[
\ddot{s} + g sin(\frac{s}{L}) = 0    
\]
Essendo questa un'equazione differenziale trascendente senza soluzione
analitica, operiamo l'approssimazione 
\[
    sin(\frac{s}{L}) \approx \frac{s}{L}
\]
sviluppando la serie di Taylor al primo ordine(la percentuale di errore 
è molto piccola per angoli piccoli). A questo punto si ottiene l'equazione
\[
    \ddot{s} + \frac{g}{L}s = 0
\]
che è la nota equazione del moto oscillatorio armonico con legge oraria:
\[
    s(t) = Acos (\omega_0 t + \varphi_0)
\]
e di pulsazione e periodo:
\[
 \omega_0 = \sqrt{\frac{g}{L}}   \qquad 
 T = \frac{2 \pi}{\omega_0} = 2\pi \sqrt{\frac{L}{g}} 
\]
La legge orari
\subsection*{Applicazione delle leggi della dinamica}
\textbf{LE FORZE ELASTICHE} \\
Le forze esercitate dalle molle sono un'esempio di forze che dipendono solo dalla posizione. Nel caso
di una molla ideale si ha che la forza di richiamo agisce solo sull'asse della molla, in verso
 opposto alla deformazione di questa ed è proporzionale all'allungamento.
Ciò viene descritto attraverso la relazione empirica seguente.
\begin{legge}[Legge di Hooke]
    \[
    \vec{F} = -kx \, \v{i}
    \]
\end{legge} 
Analizziamo il moto di una molla ideale. \\
Applicando il secondo principio della dinamica:
\[
  m \ddot{x} = -\frac{k}{m} x    
\]
Dalla risoluzione dell'equazione differenziale caratterizzante il moto oscillatorio armonico si ha:
\[
x(t) = x_0 cos(\omega_0 t + \varphi_0)    
\]
da cui:
\[
 \ddot{x}(t) = - \omega^2 x_0 cos(\omega_0 t)    
\]
Da cui si ottiene:
\[
  \omega_0 = \sqrt{\frac{k}{m}}    
\]
