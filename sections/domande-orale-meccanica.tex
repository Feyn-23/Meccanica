\section*{DOMANDE ORALE DI MECCANICA}

\subsection*{Gravitazione}
\begin{itemize}

    \item Esperimento di Cavendish con il pendolo di torsione per misurare la costante gravitazionale
    \item Dimostazione della seconda legge di Keplero con la dinamica Newtoniana ($\frac{dA}{dt}= cost$)
    \item Calcolo, attraverso la definizione di lavoro, della funzione energia potenziale di una forza 
    conservativa(per esempio la forza gravitazionale $U(r) = -G\frac{Mm}{r}$
    \item Descrizione del sistema a due corpi in coordinate polari e potenziale efficace: 
    \[
    \vec{v} = \dot{r} \v{r} + r \dot{\varphi} \v{\varphi}    
    \]
    \[
    \vec{L} = \vec{r} \cross \mu \vec{v} = \mu r^2 \dot{\varphi} \v{k}  
    \]
    \[
    U_{\text{eff}} = \frac{L^2}{2 \mu r^2} - G \frac{Mm}{r}    
    \]

\end{itemize}
\subsection*{Dinamica}
\begin{itemize}
    \item Calcolo del periodo di un pendolo semplice \\
    Le forze in gioco sono le seguenti:
    \[
     \vec{F}_p + \vec{R} = m \vec{a}    
    \]
    Il sistema che descrive le relazioni scalari delle forze in gioco:
    \[
    \begin{cases}
        -mg sin\theta = m \ddot{s} \\
        -mg cos \theta + R = m \frac{\dot{s}^2}{L}
    \end{cases}    
    \]
    Utilizzando il fatto che $\theta = \frac{s}{L}$ la prima equazione 
    diventa:
    \[
    \ddot{s} + g sin(\frac{s}{L}) = 0    
    \]
    Essendo questa un'equazione differenziale trascendente senza soluzione
    analitica, operiamo l'approssimazione 
    \[
        sin(\frac{s}{L}) \approx \frac{s}{L}
    \]
    sviluppando la serie di Taylor al primo ordine(la percentuale di errore 
    è molto piccola per angoli piccoli). A questo punto si ottiene l'equazione
    \[
        \ddot{s} + \frac{g}{L}s = 0
    \]
    che è la nota equazione del moto oscillatorio armonico con legge oraria:
    \[
        s(t) = Acos (\omega_0 t + \varphi_0)
    \]
    e di pulsazione e periodo:
    \[
     \omega_0 = \sqrt{\frac{g}{L}}   \qquad 
     T = \frac{2 \pi}{\omega_0} = 2\pi \sqrt{\frac{L}{g}} 
    \]



    \item Esaminare la dinamica di un corpo dal punto di vista del sistema di riferimento del 
    centro di massa

\end{itemize}

\subsection*{Corpi rigidi}
\begin{itemize}
    \item Dimostrazione della formula del momento angolare per un corpo rigido con
    simmetria rotazionale ($L= I \omega$)
\end{itemize}

