\subsection*{Le trasformazioni di Galileo}
Considerando due sistemi di riferimento, $S $ ed $S'$ in cui il secondo ha un moto 
di \emph{traslazione rettilinea uniforme} rispetto al primo, il vettore posizione e i principali
vettori cinematici si trasformano nel modo seguente:
\[
\begin{cases}
     \vec{r} = \vec{r}' + \vec{V}'  t' \\
     t = t'
\end{cases} 
\qquad 
\begin{cases}
     \vec{v} = \vec{v}' + \vec{V} \\
     \vec{a} = \vec{a}'
\end{cases}    
\]
Da queste trasformazioni si deduce che la velocità non è invariante e dipende dal sistema di 
riferimento:essa varia secondo la classica legge di composzione della velocità. \\
Al contrario l'accelerazione è invariante. \\
Il principio Galileiano afferma, quindi, che non è possibile distinguere due fenomeni fisici 


\subsection*{Teorema del triplo prodotto vettoriale}
\[
 \vec{a} \cross (\vec{b} \cross \vec{c}) = \vec{b} \cross (\vec{a} \cross \vec{c}) -
 \vec{c} \cross (\vec{a} \cross \vec{b})
\]
\subsection*{Derivate di vettori e versori }
\[
     \v{v} \equiv \frac{\vec{v}}{\Norm{v}}
\]
\emph{Derivata di un versore} (da mettere in sezione apposita)
Definendo un vettore $\omega = \v{u} \cross \v{n}$ con modulo $\frac{d \varphi}{dt}$
\[
 \frac{d\v{u}}{dt} = \frac{d \varphi}{dt} \v{n} = \vec{\omega} \cross \v{u}     
\]

\subsection*{Coordinate polari piane}
Un sistema di riferimento di coordinate polari piane è un sistema di coordinate 
caratterizzato da un \textbf{polo}, l'origine O, e da un \textbf{asse polare},
la semiretta orientata uscente da O. La direzione di ogni punto P è caratterizzata
da due versori:
\[
 \v{u}_r \quad (\text{versore radiale}) \qquad \text{e} \qquad \v{u}_{\theta}
 \quad  \text{(versore trasverso)}     
\]
con $\v{u}_r$ indicante la direzione del vettore direzione $\vec{r}$, che infatti si 
scrive come $\vec{r} = r \, \v{u}_r$, e con $\v{u}_{\theta}$ versore tangente alla 
circonferenza e diretto verso angoli $\theta$ crescenti ($\v{u}_r \perp \v{u}_{\theta}$).\\
Nel caso in cui la direzione dell'asse polare
sia quella del versore $\v{i}$ allora la relazione tra le coordinate polari piane (r, $\theta$)
e quelle cartesiane ($x,y$) è il seguente:
\[
\begin{cases}
     x = r cos \theta \\
     y = r sin \theta
\end{cases}
; \qquad
\begin{cases}
     r = \sqrt{x^2 + y^2} \\
     \theta = arctan \frac{y}{x}
\end{cases}     
\]
e tra i versori polari e cartesiani:
\[
\begin{cases}
     \v{u}_r = (\v{u}_r \cdot \v{i})\v{i} + (\v{u}_r \cdot \v{j})\v{j} =
     cos \theta\v{i} + sin \theta \v{j} \\
     \v{u}_{\theta}= (\v{u}_{\theta} \cdot \v{i})\v{i} + (\v{u}_{\theta} \cdot \v{j})\v{j} =
     -sin \theta \v{i} + cos \theta \v{j}
\end{cases}     
\]
Il vettore direzione $\vec{r} = \v{u}_r$ in relazione alle coordinate cartesiane:
\[
     \vec{r} = r cos \theta \v{i} + r sin \theta \v{j}     
\]
Mentre il vettore $\vec{\varphi}$ sempre tangente alla curva con direzione $\v{u}_{\theta}$
(in un moto circolare corrisponde alla velocità tangenziale): 
\[
    \vec{\varphi} = - \varphi sin \theta + \varphi cos \theta     
\]
\subsection*{Coordinate polari sferiche}
